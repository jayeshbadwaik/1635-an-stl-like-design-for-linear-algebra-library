\documentclass[oneside,11pt,a4paper]{jbarticle}

\addbibresource{library.bib}

\subject{}
\subtitle{}
\title{D1635R0 : An STL-like Design for Linear Algebra Library}
\author{Jayesh Badwaik}

\begin{document}
\maketitle[\value{page}]
\begin{abstract}
  A linear algebra library is being currently being considered for
  standardization.  In this paper, we present some of the techniques that will
  allow us to design a library that will allow interoperability with other
  suitably-designed linear algebra libraries. The design is inspired from the
  design of the standard template library in C++ and its ability to allow the
  user to use custom types with standard algorithms and vice versa.
\end{abstract}

\section{Introduction}
For some years now, there have been efforts to add a linear algebra library in
the C++ standard library.  One of the motivations has been to provide
common vocabulary types to allow different code bases to use each other without
having to make intermediate copies. The other factor which is generally
important is to recognize the sparsity in the data structures and . The way
The focus of those papers mainly
deals with the types and algorithms which would be useful in a standardized
linear algebra library.


The named functions themselves are often not a problem because one almost
always, one can choose the correct algorithms based on the context. Furthermore,
choosing the correct algorithm is straightforward as well, often as simple as
\begin{codecpp}{Calling a Custom Function}
  std::some_algorithm(custom_execution_policy, arguments...);
  custom_lib::custom_algorithm(custom_execution_policy, arguments...);
\end{codecpp}

The problems arise when one tries to deal with operators. For example, how
should the expression \inlinecode{a+b}

One of the major movements in the past couple of standards has been a move
towards using ranges and views. In addition to this, the linear algebra
libraries in C++ have often made use of expression templates.

\subsection{What this paper is not?}

\section{Owning Types}
\section{Engine-Aware Types}
\section{Engine-Based Operators}
\section{Some Musings on Design Decisions}
\subsection{Offloading Computation}
\subsection{Use of ADL in Detection of Correct Operator Namespace}
Justification and Pitfalls (Could also be thought of as good design decisions.)
Good practice is not to write operators for linear algebra types but instead
write them for the engines.
\section{Example}
\section{Acknowledgement}
A lot of ideas have been gathered by participating in the SG15 Linear Algebra
SIG discussions, reading the codebases like Eigen, MTL, Blaze and others etc.
While I have tried to credit the ideas wherever necessary, if you find your
ideas being used in the code without credit, I apologize in advance. Also, I
would be grateful if you could bring them to my attention so that I could write
them as such.

The idea of offloading the computations to an engine is inspired from
\cite{GuyDavidson2018}. The idea to use execution policy is inspired from MTL
\incomplete{Check this?}.


\printbibliography
\end{document}
